\documentclass{standalone}

\usepackage{tikz}


\begin{document}

% draw a commit node
\newcommand{\commitnode}[5]{
    {
    \node[#1, fill=white!#2, minimum width= 2.4cm, minimum height=1.2cm, text width=2.2cm, align=left] (cnew)   
    (#5) at #3  [draw] {\vbox to 1cm{\vfill \hbox{\textsf{\tiny{#4}}}}};
    }
}

\newcommand{\commitgroup}[3]{
    \begin{scope}[shift={#1}, scale=#2,
        every node/.append style={transform shape}]
        \commitnode{}{100}{(0.3,-0.7)}{...........}{others_#3}
        \commitnode{}{100}{(0.2,-0.4)}{54a78ac2...}{c1_#3}
        \commitnode{}{100}{(0.1,-0.1)}{deadbeef...}{c2_#3}
        \commitnode{}{100}{(0.0, 0.2)}{eaf329aa...}{c3_#3}
        \commitnode{dotted}{100}{(-0.1,0.5)}{\tiny{$<$new commit$>$}}{cnew_#3};
    \end{scope}
}

   % TODO : convert into beamer

  \begin{tikzpicture}[outer sep=1]
      \path [use as bounding box] (-1.0,-1.0) rectangle (12,5);

    % this is the only version of your file that applications outside of git see.
    \node[text width=1.5cm, align=center] (wt)   at (0,2)  [draw] {working tree};

    % when git comes into play, many more versions become available
    \node[text width=1.5cm, align=center] (git)   at (4.5,4)  {\textsf{.git}};
    \draw [dotted] (1.5,4.5) -- (1.5,0); 

    % there is the stagin area
    \node[text width=2.0cm, align=center] (idx)   at (3,2)  [draw] {index (staging area)};


      \commitgroup{(6.0,2.0)}{1.0}{local}

    % git add copies the version of a file to the index
    % that it, it overwrites the version in the index.

    \draw[->, auto=right] (wt) edge [ bend right=45 ] node[pos=0.5] {\tiny{add}} (idx) ;

    % new commit


    \draw[->, auto=left] (idx) edge [ bend left=45 ] node[pos=0.25] {\tiny{commit}} (cnew_local) ;


    % there are also remotes!

      \draw [dotted] (8.0,4.5) -- (8.0,0); 
      \node[text width=1.5cm, align=center] (origin)   at (9.5,4)  {\textsf{remote}};
      \commitgroup{(10.0,2.5)}{0.3}{remote}

    % new commits are pushed to a remote with push

      \draw[->] (others_local) edge [out=30, in=150]  (others_remote);
      \draw[->] (c1_local) edge [out=30, in=150]  (c1_remote);
      \draw[->] (c2_local) edge [out=30, in=150]  (c2_remote);
      \draw[->] (c3_local) edge [out=30, in=150]  (c3_remote);
      \draw[->, auto=left] (cnew_local) edge [out=30, in=150] node [pos=0.5] {\tiny{push}} (cnew_remote);

      \draw[->, auto=left] (others_remote) edge [in=330, out=210] node [pos=0.5] {\tiny{fetch}}  (others_local);
      \draw[->] (c1_remote) edge [in=330, out=210]  (c1_local);
      \draw[->] (c2_remote) edge [in=330, out=210]  (c2_local);
      \draw[->] (c3_remote) edge [in=330, out=210]  (c3_local);
      \draw[->] (cnew_remote) edge [in=330, out=210]  (cnew_local);


    % one can also commit directly from the worktree

    \draw[->, auto=left] (wt) edge [ bend left=45 ] (cnew_local) ;

    % this affects also the index
    \draw[->, auto=left] (wt) edge [ bend left=45 ] node[pos=0.5] {\tiny{commit -a }} (idx) ;

    % there's also restore in its different incarnations
      \draw[->, auto=left] (c1_local) edge [bend left = 45 ] node[pos=0.5] {\tiny{restore}} (idx);
      \draw[->, auto=left] (c1_local) edge [bend left = 45 ] node[pos=0.5] {\tiny{restore}} (wt);



  \end{tikzpicture}
\end{document}
